\documentclass[10pt,a4paper]{article}
\usepackage{amsmath,enumitem,fancyhdr,mathtools,graphicx}
\usepackage[margin=2cm]{geometry}
\newcommand{\twovec}[2]{\ensuremath{\begin{pmatrix}{#1}\\{#2}\end{pmatrix}}}
\newcommand{\V}[1]{\vec{\mathbf{#1}}}

\DeclarePairedDelimiter{\norm}{\big\lVert}{\big\rVert}

\pagestyle{fancy}
\fancyhead[L]{Visit Gym Neufeld}
\fancyhead[C]{Exercise Sheet (Math)}
\fancyhead[R]{1 July, 2025}

\begin{document}
\section{Inner products}
We describe points in 2D-space by their \(x\)- and \(y\)-coordinate.
For example: \(\twovec{a}{b}\) is the point with \(x\)-coordinate \(a\) and \(y\)-coordinate \(b\).
If we have two points \(\V{v} = \twovec{a}{b}\) and \(\V{w} = \twovec{c}{d}\),
	the ``inner product'' of \(\V{v}\) and \(\V{w}\) is:
\[\V{v} \cdot \V{w} = a \times c \;+\; b \times d.\]

In the exercises below, we will see why this definition is nice and meaningful.

\begin{enumerate}[label=(\arabic*)]
\item Simplify the inner product in the special case when the first vector is \( \V{v} = \twovec{1}{0}\).
	Then: \(\twovec{1}{0} \cdot \twovec{c}{d} = \ldots\)
	\begin{figure}[h!]
		\centering\fbox{\includegraphics[width=.5\textwidth]{tikz/x-axis.pdf}}
	\end{figure}

\item How does this relate to \(\V{w}\)? If \(\V{v} \cdot \V{w} = 0\), what is the angle between \(\V{v}\) and \(\V{w}\)?

\item Now look at the inner product with \( \V{v} = \twovec{a}{0}\).
	In this case, \(\V{v} \cdot \V{v} = a^2\).

\item We can rotate a point \(\V{v} = \twovec{a}{b}\) counter-clockwise by an angle of \(\theta\)
	as follows: \(\V{v}\,' = \twovec{a \cos(\theta) - b \sin(\theta)}{a \sin(\theta) + b \cos(\theta)}\).\\[-1em]
	\begin{itemize}
	\item Verify the equation for \(\theta = 90^\circ\).
		Then: \(\V{v}\,' = \twovec{\quad\ldots\quad}{\quad\ldots\quad}\).

	\item Write \(\V{w}\,'\) for the rotation of \(\V{w}\) by an angle of \(\theta\).
		Show: \(\V{v} \cdot \V{w} = \V{v}\,' \cdot \V{w}\,'\).%
		\footnote{Hint: recall \(\cos^2(\theta) + \sin^2(\theta) = 1\).}
	\end{itemize}

	\begin{figure}[h!]
		\centering\fbox{\includegraphics[width=.5\textwidth]{tikz/givens.pdf}}
	\end{figure}
\item Describe the angle \(\theta\) (using \(a, b\) and \(\sin/\cos/\tan\)) that rotates \(\V{v} = \twovec{a}{b}\) to \(\V{v}\,' = \twovec{x}{0}\) for some \(x > 0\).
	What is \(x\)?

\item The length of a vector \(\V{v} = \twovec{a}{b}\) is given by \(\norm{\V{v}} = \sqrt{a^2 + b^2}\).
	Prove: \(\norm{\V{v}}^2 = \V{v} \cdot \V{v}\).

\item Suppose \(\V{v}\) and \(\V{w}\) have an angle \(\theta\) between them.
	Prove (using the previous exercises):\footnote{Hint: use (5) to rotate \(\V{v}\) first, and rotate \(\V{w}\) similarly.}
	\[\V{v} \cdot \V{w} = \cos(\theta) \times \norm{\V{v}} \times \norm{\V{w}}.\]
\end{enumerate}

\clearpage
\section{Parallelograms}
% \det(v . w) is invariant.
Recall that a basis gives a tiling.
We will relate the area to the basis.
First assume \(\V{v}\) is on the \(x\)-axis.

\begin{figure}[h!]
	\centering\fbox{\includegraphics[width=.5\textwidth]{tikz/parallelogram.pdf}}
\end{figure}

\begin{enumerate}[label=(\arabic*)]
\item Show that the parallelogram \(O, \V{v}, \V{v}+\V{w}, \V{w}\), and the rectangle \(O, A, \V{w}, C\) have equal area.\footnote{Hint: go from one to the other by adding a triangle and removing a triangle (of same area).}
\item What is the area?
\end{enumerate}

\section{Gram--Schmidt and projections}

\section{Lagrange reduction}

\end{document}
