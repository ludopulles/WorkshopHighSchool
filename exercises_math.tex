% chktex-file 8
\documentclass[10pt,a4paper]{article}
\usepackage{amsmath,amssymb,enumitem,fancyhdr,mathtools,graphicx,wrapfig}
\usepackage[margin=2cm]{geometry}
\newcommand{\twovec}[2]{\ensuremath{\begin{pmatrix}{#1}\\{#2}\end{pmatrix}}}

\NewCommandCopy{\oldvec}{\vec} % https://tex.stackexchange.com/questions/47351/can-i-redefine-a-command-to-contain-itself
\renewcommand{\vec}[1]{\oldvec{\mathbf{#1}}}
\newcommand{\RR}{\mathbb{R}}

\DeclarePairedDelimiter{\norm}{\big\lVert}{\big\rVert}
\setlist[enumerate,1]{labelindent=\parindent,label=\textbf{(\alph*)}}
\setlist[enumerate,2]{labelindent=\parindent,label=\textbf{(\arabic*)}}

\pagestyle{fancy}
\fancyhead[L]{Visit Gym Neufeld}
\fancyhead[C]{Exercise Sheet (Math)}
\fancyhead[R]{1 July, 2025}

\begin{document}
\section{Inner products}
We describe points in 2D-space by their \(x\)- and \(y\)-coordinate.
For example: \(\twovec{a}{b}\) is the point with \(x\)-coordinate \(a\) and \(y\)-coordinate \(b\).
If we have two points \(\vec{v} = \twovec{a}{b}\) and \(\vec{w} = \twovec{c}{d}\),
	the ``inner product'' of \(\vec{v}\) and \(\vec{w}\) is:
\[\vec{v} \cdot \vec{w} = a \times c \;+\; b \times d.\]
In the exercises below, we will see why this definition is nice and meaningful.

\begin{enumerate}
\item Simplify the inner product in the special case when the first vector is \( \vec{v} = \twovec{1}{0}\).
	Then: \(\twovec{1}{0} \cdot \twovec{c}{d} = \ldots\)
	\begin{figure}[h!]
		\centering\fbox{\includegraphics[width=.5\textwidth]{tikz/x-axis.pdf}}
	\end{figure}

\item How does this relate to \(\vec{w}\)? If \(\vec{v} \cdot \vec{w} = 0\), what is the angle between \(\vec{v}\) and \(\vec{w}\)?

\item Now look at the inner product with \( \vec{v} = \twovec{a}{0}\).
	In this case, \(\vec{v} \cdot \vec{v} = a^2\).

\item We can rotate a point \(\vec{v} = \twovec{a}{b}\) counter-clockwise by an angle of \(\theta\)
	as follows: \(\vec{v}\,' = \twovec{a \cos(\theta) - b \sin(\theta)}{a \sin(\theta) + b \cos(\theta)}\).\\[-1em]
	\begin{enumerate}
	\item Verify the equation for \(\theta = 90^\circ\).
		Then: \(\vec{v}\,' = \twovec{\quad\ldots\quad}{\quad\ldots\quad}\).

	\item Write \(\vec{w}\,'\) for the rotation of \(\vec{w}\) by an angle of \(\theta\).
		Show: \(\vec{v} \cdot \vec{w} = \vec{v}\,' \cdot \vec{w}\,'\).%
		\footnote{Hint: recall \(\cos^2(\theta) + \sin^2(\theta) = 1\).}
	\end{enumerate}

	\begin{figure}[h!]
		\centering\fbox{\includegraphics[width=.5\textwidth]{tikz/givens.pdf}}
	\end{figure}
\item Describe the angle \(\theta\) (using \(a, b\) and \(\sin/\cos/\tan\)) that rotates \(\vec{v} = \twovec{a}{b}\) to \(\vec{v}\,' = \twovec{x}{0}\) for some \(x > 0\).
	What is \(x\)?

\item The length of a vector \(\vec{v} = \twovec{a}{b}\) is given by \(\norm{\vec{v}} = \sqrt{a^2 + b^2}\).
	Prove: \(\norm{\vec{v}}^2 = \vec{v} \cdot \vec{v}\).

\item Suppose the angle between \(\vec{v}\) and \(\vec{w}\) is \(\theta\).
	Prove (using the previous exercises):\footnote{Hint: use (5) to rotate \(\vec{w}\) with the same angle that rotates \(\vec{v}\) onto the \(x\)-axis.}
	\[\vec{v} \cdot \vec{w} = \cos(\theta) \times \norm{\vec{v}} \times \norm{\vec{w}}.\]
\end{enumerate}

\clearpage
\section{Parallelograms}
\begin{wrapfigure}[5]{r}[1em]{.3\textwidth}
		\vspace*{-3em}
		\fbox{\includegraphics[width=.3\textwidth]{tikz/parallelogram.pdf}}
		\caption{One tile}
\end{wrapfigure}

Recall that a basis for a lattice gives a tiling of space.

For~\ref{item1},~\ref{item2} and~\ref{item3}, assume \(\vec{v}\) is on the \(x\)-axis.
First, we relate the area of the parallelogram to the basis.
\begin{enumerate}
	\item\label{item1} Show: the parallelogram \(O, \vec{v}, \vec{v}+\vec{w}, \vec{w}\), and the rectangle \(O, A, \vec{w}, C\) have equal
		area.\footnote{Hint: go from one to the other by adding a triangle and removing a triangle (of same area).}
	\item\label{item2} Show: the area is independent of \(c\),~i.e.~replacing \(\vec{w}\) by \(\vec{w} + z \vec{v}\) does not change the area for any \(z\)!
	\item\label{item3} What is the area in terms of \(x\) and \(d\)?
\end{enumerate}

\noindent
Now consider the general case of \(\vec{v} = \twovec{a}{b}\) and \(\vec{w} = \twovec{c}{d}\).
\begin{figure}[h!]
	\centering\fbox{\includegraphics[width=.6\textwidth]{tikz/tiling.pdf}}
	\caption{Tiling of 2D-space by the lattice basis \([\vec{v}, \vec{w}]\)}\label{fig:tiling}
\end{figure}

\begin{enumerate}[resume*]
	\item Using~\ref{item2}, slide \(\vec{v}\) to a point on the \(x\)-axis.
		Show: the parallelogram has area \(ad - bc\).\footnote{Hint: first determine for which value of \(z\), the point \(\vec{v} - z \vec{w}\) is on the \(x\)-axis. Then determine the \(x\)-coordinate of this point.}

	\item How does the area change when you:
		\begin{enumerate}
		\item swap \(\vec{v}\) and \(\vec{w}\)?
		\item replace \(\vec{w}\) by \(\vec{w} + \vec{v}\)?
		\item multiply \(\vec{v}\) and \(\vec{w}\) by \(2\)?
		\end{enumerate}
	\item Conclude from (1) and (2) that the area stays the same during lattice reduction.

	\item
		Show: the circle in Figure~\ref{fig:tiling-np} has radius \(r = \frac12 \norm{\vec{v}}\). \\
		Show: the circle in Figure~\ref{fig:tiling} has radius \(r' = \frac12 \norm{\vec{v}} \times \sin(\theta) \leq r\), where \(\theta\) is the angle between \(\vec{v}\) and \(\vec{w}\). \\
\end{enumerate}
\begin{figure}[h!]
	\centering\fbox{\includegraphics[width=.6\textwidth]{tikz/tiling-NP.pdf}}
	\caption{A different tiling of 2D-space, using a rectangular tile.}\label{fig:tiling-np}
\end{figure}
\noindent
\textbf{Encryption with lattices.}
With the secret key (the good basis \([\vec{v},\vec{w}]\)) only you can decrypt ciphertexts sent by others!
Namely, a ciphertext is \(\vec{t} = \vec{m} + \vec{e}\),
	where \(\vec{m} = a \vec{v} + b \vec{w}\) (\(a, b\) integer) corresponds to a message,
	and \(\vec{e}\) is a short vector.
With the secret key, you can find the tile containing \(\vec{t}\) and recover \(\vec{m}\), \emph{if} \(\norm{\vec{e}}\) is small enough.

\begin{enumerate}[resume*]
	\item Show: decrypting with the tile in Fig.~\ref{fig:tiling}
	works if \(\norm{\vec{e}} \leq r'\),
	and with the tile in Fig.~\ref{fig:tiling-np}, it works if \(\norm{\vec{e}} \leq r\).
	Which of the two tilings works best at decryption?
\end{enumerate}

\clearpage
\section{Gram--Schmidt and projections}

Instead of 2D-space (\(\RR^2\)), in this exercise we will look at \(n\)-dimensional space (\(\RR^n\)).
Now, vectors are described by \(n\) coordinates:
\[
	\vec{v} = \begin{pmatrix} v_1 \\ v_2 \\ \cdots \\ v_n \end{pmatrix}.
\]
We still have the inner product, which is:
\(\vec{v} \cdot \vec{w} = v_1 \times w_1 + v_2 \times w_2 + \ldots v_n \times w_n\).

\section{Lagrange reduction}

\end{document}
