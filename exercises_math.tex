\documentclass[12pt,a4paper]{article}
\usepackage{amsmath,amssymb,amsthm,enumitem,setspace}
\usepackage[top=15mm]{geometry}

\newcommand{\twovec}[2]{\ensuremath{\begin{pmatrix}{#1}\\{#2}\end{pmatrix}}}

\author{Ludo Pulles}
\date{1 July, 2025}
\title{Exercise Sheet --- Math}

\begin{document}
\maketitle
\onehalfspacing%

\section{Vectors and inner products}

\subsection{Dimension 2}
We describe points in 2d-space by their \(x\)- and \(y\)-coordinate.
We write \(\displaystyle\twovec{x}{y}\) for a point with \(x\)-coordinate \(x\) and \(y\)-coordinate \(y\).
If we have two points, \(\displaystyle\vec{v} = \twovec{a}{b}\) and \(\displaystyle\vec{w} = \twovec{c}{d}\) the ``dot product'' of \(\vec{v}\) and \(\vec{w}\) is: \(\vec{v} \cdot \vec{w} = ac + bd\).
In the following exercises, we see why this definition is nice and meaningful.

\begin{enumerate}[label=(i)]
\item A very special case of the dot product is when the first vector is \(\displaystyle \vec{v} = \twovec{1}{0}\).
	Simplify the ``dot product'': \(\twovec{1}{0} \cdot \twovec{c}{d}\) in this case.
\end{enumerate}

\end{document}
